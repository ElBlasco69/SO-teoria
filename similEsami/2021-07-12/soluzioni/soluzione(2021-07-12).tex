\documentclass{article}
\usepackage[table]{xcolor}
\usepackage[margin=1.5cm]{geometry}
\usepackage{caption}
\usepackage{mathtools}
\usepackage{amsfonts}
\usepackage{hyperref}
\usepackage{listings}
\lstset{
  basicstyle=\ttfamily,
  mathescape
}
\title{Esame 2018/09/10}
\author{Blascovich Alessio}
\date{}

\begin{document}
    \maketitle
    \section*{Domanda 1}
    \textbf{MAX PUNTI:} 6\\
    \textbf{Domanda:}


    Spiegare nel dettaglio cos'è e come viene usata la memoria TLB.\\
    \textbf{Soluzione:}


    \emph{TLB}(Translation look-aside buffer) è una specia di cache per tradurre gli indirizzi di memoria in modo rapido, senza dover scorrere la tabella delle pagine.\\
    Serve per evitare il problema del doppia accesso in memoria dato sal salvataggio in memoria della tabella delle pagine.\\
    La/il TLB confronta contemporaneamente i numeri di tutte le pagine con quello fornito nell'indirizzo logico, se lo trova(TLB hit) lo concatena con l'offset, mentre se non lo trova(TLB miss) deve scannerizzare la tabella delle pagine.\\
    La/il TLB essendo una forma di cache ha un rapporto dimensioni/costo molto elevato quindi al suo interno vengono memorizzate solo poche entry che verranno ripulite ad ogni context switch per evitare di mappare indirizzi errati.
    \section*{Domanda 2}
    \textbf{MAX PUNTI:} 6\\
    \textbf{Domanda:}


    Si consideri di avere i processi $P_A$, $P_B$, $P_C$ e $P_D$ con le risorse $R_1$, $R_2$, $R_3$, $R_4$ e $R_5.$\\
    Sono illustrate nelle matrici sottostanti le risorse allocate e quelle massime per ogni processo:\\
    \begin{minipage}[c]{0.5\textwidth}
        \captionof*{table}{\emph{ALLOC}}
        \centering
        \begin{tabular}{|c|c|c|c|c|c|}
            \hline
            & $R_1$ & $R_2$ & $R_3$ & $R_4$ & $R_5$\\
            \hline
            $P_A$ & 1 & 0 & 2 & 1 & 1\\
            \hline
            $P_B$ & 2 & 0 & 1 & 1 & 1\\
            \hline
            $P_C$ & 1 & 1 & 0 & 1 & 0\\
            \hline
            $P_D$ & 1 & 1 & 1 & 1 & 0\\
            \hline
        \end{tabular}
    \end{minipage}
    \begin{minipage}[c]{0.5\textwidth}
        \captionof*{table}{\emph{MAX}}
        \centering
        \begin{tabular}{|c|c|c|c|c|c|}
            \hline
            & $R_1$ & $R_2$ & $R_3$ & $R_4$ & $R_5$\\
            \hline
            $P_A$ & 1 & 1 & 2 & 1 & 3\\
            \hline
            $P_B$ & 2 & 2 & 2 & 1 & 1\\
            \hline
            $P_C$ & 2 & 1 & 3 & 1 & 0\\
            \hline
            $P_D$ & 1 & 1 & 2 & 2 & 1\\
            \hline
        \end{tabular}
    \end{minipage}\\
    Se il vettore delle risorse è A=(0,0,$x$,1,1), quale valore deve avere $x$ per rendere lo stato sicuro?(mostrare anche il procedimento che ha portato alla scelta di x)\\
    \textbf{Soluzione:}


    Provo ad usare l'algoritmo del banchiere \href{https://www.youtube.com/watch?app=desktop&v=LcspXXBImEo}{spiegazione}, quindi devo creare la matrice \emph{Need}, definita come \emph{Need}[i]=\emph{MAX}[i]-\emph{ALLOC}[i].
    \begin{center}
        \emph{Need}\\
        \begin{tabular}{|c|c|c|c|c|c|}
            \hline
            & $R_1$ & $R_2$ & $R_3$ & $R_4$ & $R_5$\\
            \hline
            $P_A$ & 0 & 1 & 0 & 0 & 2\\
            \hline
            $P_B$ & 0 & 2 & 1 & 0 & 0\\
            \hline
            $P_C$ & 1 & 0 & 3 & 0 & 0\\
            \hline
            $P_D$ & 0 & 0 & 1 & 1 & 1\\
            \hline
        \end{tabular}
    \end{center}
    Ora devo trovare una $x$ per cui vale $\forall i \in [A,\dots,D]$ : \emph{Need}[i][]$\le$A[], pongo work[]=A[].\\
    \begin{itemize}
        \item $x$=0 $\Rightarrow$ work=(0,0,0,1,1)
            \begin{itemize}
                \item (0,1,0,0,2)$\le$(0,0,0,1,1) $\Rightarrow$ \verb+false+.
                \item (0,2,1,0,0)$\le$(0,0,0,1,1) $\Rightarrow$ \verb+false+.
                \item (1,0,3,0,0)$\le$(0,0,0,1,1) $\Rightarrow$ \verb+false+.
                \item (0,0,1,1,1)$\le$(0,0,0,1,1) $\Rightarrow$ \verb+false+.
            \end{itemize}
            Se da dal primo ciclo non incrmeneto nulla posso passare alla prossima $x$.
        \item $x$=1 $\Rightarrow$ work=(0,0,1,1,1)
            \begin{itemize}
                \item (0,1,0,0,2)$\le$(0,0,1,1,1) $\Rightarrow$ \verb+false+.
                \item (0,2,1,0,0)$\le$(0,0,1,1,1) $\Rightarrow$ \verb+false+.
                \item (1,0,3,0,0)$\le$(0,0,1,1,1) $\Rightarrow$ \verb+false+.
                \item (0,0,1,1,1)$\le$(0,0,1,1,1) \checkmark $\Rightarrow$ work[]=work[]+\emph{ALLOC}[$P_D$]=(1,1,2,2,1).
            \end{itemize}
            Ma anche ripetendo non sono in uno stato safe quindi passo alla prossima $x$.
        \item $x$=2 $\Rightarrow$ work=(0,0,2,1,1)
        \begin{itemize}
            \item (0,1,0,0,2)$\le$(0,0,2,1,1) $\Rightarrow$ \verb+false+.
            \item (0,2,1,0,0)$\le$(0,0,2,1,1) $\Rightarrow$ \verb+false+.
            \item (1,0,3,0,0)$\le$(0,0,2,1,1) $\Rightarrow$ \verb+false+.
            \item (0,0,1,1,1)$\le$(0,0,2,1,1) \checkmark $\Rightarrow$ work[]=work[]+\emph{ALLOC}[$P_D$]=(1,1,3,2,1).
        \end{itemize}
        Adesso ripeto il ciclo escludendo $P_D$ con work[]=(1,1,3,2,1).
        \begin{itemize}
            \item (0,1,0,0,2)$\le$(1,1,3,2,1) $\Rightarrow$ \verb+false+.
            \item (0,2,1,0,0)$\le$(1,1,3,2,1) $\Rightarrow$ \verb+false+.
            \item (1,0,3,0,0)$\le$(1,1,3,2,1) \checkmark $\Rightarrow$ work[]=work[]+\emph{ALLOC}[$P_C$]=(2,2,3,3,1).
        \end{itemize}
        Adesso ripeto il ciclo escludendo $P_D$ e $P_C$ con work[]=(2,2,3,3,1).
        \begin{itemize}
            \item (0,1,0,0,2)$\le$(2,2,3,3,1) $\Rightarrow$ \verb+false+.
            \item (0,2,1,0,0)$\le$(2,2,3,3,1) \checkmark $\Rightarrow$ work[]=work[]+\emph{ALLOC}[$P_B$]=(4,2,4,4,2).
        \end{itemize}
        Adesso ripeto il ciclo escludendo $P_D$, $P_C$ e $P_B$ con work[]=(4,2,4,4,2).
        \begin{itemize}
            \item (0,1,0,0,2)$\le$(4,2,4,4,2) \checkmark $\Rightarrow$ work[]=work[]+\emph{ALLOC}[$P_A$]=(5,2,6,5,3)
        \end{itemize}
    \end{itemize}
    Iterando l'algoritmo ho scoperto che il più piccolo vettore A è (0,0,2,1,1) quindi con $x$=2.
    \section*{Domanda 3}
    \textbf{MAX PUNTI:} 6\\
    \textbf{Domanda:}


    Spiegare nel dettaglio come funzionano gli i-node e come contengono informazioni.\\
    \textbf{Soluzione:}


    Gli i-node sono usati nel mondo UNIX per rappresentare i file, in un i-node sono memorizzati i metadati di un file e fa anche da indice per il file.\\
    Gli i-node sono gestiti dal O.S. quindi sono permanentemente memorizzati in una porzione riservata del disco(solitamente all'inizio).\\
    Dopo i metadati nell'i-node sono memorizzati anche:
    \begin{itemize}
        \item 10 puntatori per l'accesso diretto al file.
        \item 1 puntatore ad una tabella di indici di 1° livello.
        \item 1 puntatore ad una tabella di indici di 2° livello.
        \item 1 puntatore ad una tabella di indici di 3° livello.
    \end{itemize}
    \section*{Domanda 4}
    \textbf{MAX PUNTI:} 6\\
    \textbf{Domanda:}


    Data la stringa di frame di memoria: 1,3,2,4,1,4,3,5,1,3,2,4,5,6,1,3,4,6,4,1,2,6,1,3,4,2.\\
    Determinare il numero di page fault dato dagli algoritmi \emph{FIFO}, \emph{LRU} e \emph{Ideale} ipotizzando si avere 3 frame a disposizione inizialmente vuoti.\\
    \textbf{Soluzione:}


    Si consideri la sigla pf come l'acronimo di page-fault.
    \begin{itemize}
        \item \emph{FIFO}:\\
            Il metodo \emph{FIFO} gestisce la memeoria come una semplice coda, il primo segmaneto ad essere entrato è il primo ad essere fatto uscire.\\
            \begin{tabular}{|c|c|c|}
                \hline
                1 & null & null\\
                \hline
            \end{tabular}
            $\xRightarrow{\text{pf}=1}$
            \begin{tabular}{|c|c|c|}
                \hline
                    1 & 3 & null\\
                \hline
            \end{tabular}
            $\xRightarrow{\text{pf}=2}$
            \begin{tabular}{|c|c|c|}
                \hline
                1 & 3 & 2\\
                \hline
            \end{tabular}
            $\xRightarrow{\text{pf}=3}$
            \begin{tabular}{|c|c|c|}
                \hline
                4 & 3 & 2\\
                \hline
            \end{tabular}
            $\xRightarrow{\text{pf}=4}$
            \begin{tabular}{|c|c|c|}
                \hline
                4 & 1 & 2\\
                \hline
            \end{tabular}
            $\xRightarrow{\text{pf}=5}$
            \begin{tabular}{|c|c|c|}
                \hline
                4 & 1 & 2\\
                \hline
            \end{tabular}
            $\xRightarrow{\text{pf}=5}$
            \begin{tabular}{|c|c|c|}
                \hline
                4 & 1 & 3\\
                \hline
            \end{tabular}
            $\xRightarrow{\text{pf}=6}$
            \begin{tabular}{|c|c|c|}
                \hline
                5 & 1 & 3\\
                \hline
            \end{tabular}
            $\xRightarrow{\text{pf}=7}$
            \begin{tabular}{|c|c|c|}
                \hline
                5 & 1 & 3\\
                \hline
            \end{tabular}
            $\xRightarrow{\text{pf}=7}$
            \begin{tabular}{|c|c|c|}
                \hline
                5 & 1 & 3\\
                \hline
            \end{tabular}
            $\xRightarrow{\text{pf}=7}$
            \begin{tabular}{|c|c|c|}
                \hline
                5 & 2 & 3\\
                \hline
            \end{tabular}
            $\xRightarrow{\text{pf}=8}$
            \begin{tabular}{|c|c|c|}
                \hline
                5 & 2 & 4\\
                \hline
            \end{tabular}
            $\xRightarrow{\text{pf}=9}$
            \begin{tabular}{|c|c|c|}
                \hline
                5 & 2 & 4\\
                \hline
            \end{tabular}
            $\xRightarrow{\text{pf}=9}$
            \begin{tabular}{|c|c|c|}
                \hline
                6 & 2 & 4\\
                \hline
            \end{tabular}
            $\xRightarrow{\text{pf}=10}$
            \begin{tabular}{|c|c|c|}
                \hline
                6 & 1 & 4\\
                \hline
            \end{tabular}
            $\xRightarrow{\text{pf}=11}$
            \begin{tabular}{|c|c|c|}
                \hline
                6 & 1 & 3\\
                \hline
            \end{tabular}
            $\xRightarrow{\text{pf}=12}$
            \begin{tabular}{|c|c|c|}
                \hline
                4 & 1 & 3\\
                \hline
            \end{tabular}
            $\xRightarrow{\text{pf}=13}$
            \begin{tabular}{|c|c|c|}
                \hline
                4 & 6 & 3\\
                \hline
            \end{tabular}
            $\xRightarrow{\text{pf}=14}$
            \begin{tabular}{|c|c|c|}
                \hline
                4 & 6 & 3\\
                \hline
            \end{tabular}
            $\xRightarrow{\text{pf}=14}$
            \begin{tabular}{|c|c|c|}
                \hline
                4 & 6 & 1\\
                \hline
            \end{tabular}
            $\xRightarrow{\text{pf}=15}$
            \begin{tabular}{|c|c|c|}
                \hline
                4 & 6 & 1\\
                \hline
            \end{tabular}
            $\xRightarrow{\text{pf}=15}$
            \begin{tabular}{|c|c|c|}
                \hline
                4 & 6 & 1\\
                \hline
            \end{tabular}
            $\xRightarrow{\text{pf}=15}$
            \begin{tabular}{|c|c|c|}
                \hline
                4 & 6 & 1\\
                \hline
            \end{tabular}
            $\xRightarrow{\text{pf}=15}$
            \begin{tabular}{|c|c|c|}
                \hline
                3 & 6 & 1\\
                \hline
            \end{tabular}
            $\xRightarrow{\text{pf}=16}$
            \begin{tabular}{|c|c|c|}
                \hline
                3 & 4 & 1\\
                \hline
            \end{tabular}
            $\xRightarrow{\text{pf}=17}$
            \begin{tabular}{|c|c|c|}
                \hline
                3 & 6 & 2\\
                \hline
            \end{tabular}
            $\xRightarrow{\text{pf}=18}$
            Termino con un totale di 18 page fault.
        \item \emph{LRU}:\\
            L'algoritmo \emph{LRU} tiene associato ad ogni frame il clock della CPU nell'istante in cui quel frame è stato usato per l'ultima volta, verrà tolto il frame che non viene usato da più tempo.\\
            Nella prima riga sono indicati gli istanti, mentre nella seconda il numero del frame.\\
            \begin{tabular}{|c|c|c|}
                \hline
                1 & 0 & 0\\
                \hline
                1 & null & null\\
                \hline
            \end{tabular}
            $\xRightarrow{\text{pf}=1}$
            \begin{tabular}{|c|c|c|}
                \hline
                1 & 2 & 0\\
                \hline
                1 & 3 & null\\
                \hline
            \end{tabular}
            $\xRightarrow{\text{pf}=2}$
            \begin{tabular}{|c|c|c|}
                \hline
                1 & 2 & 3\\
                \hline
                1 & 3 & 2\\
                \hline
            \end{tabular}
            $\xRightarrow{\text{pf}=3}$
            \begin{tabular}{|c|c|c|}
                \hline
                4 & 2 & 3\\
                \hline
                4 & 3 & 2\\
                \hline
            \end{tabular}
            $\xRightarrow{\text{pf}=4}$
            \begin{tabular}{|c|c|c|}
                \hline
                4 & 5 & 3\\
                \hline
                4 & 1 & 2\\
                \hline
            \end{tabular}
            $\xRightarrow{\text{pf}=5}$
            \begin{tabular}{|c|c|c|}
                \hline
                6 & 5 & 3\\
                \hline
                4 & 1 & 2\\
                \hline
            \end{tabular}
            $\xRightarrow{\text{pf}=5}$
            \begin{tabular}{|c|c|c|}
                \hline
                6 & 5 & 7\\
                \hline
                4 & 1 & 3\\
                \hline
            \end{tabular}
            $\xRightarrow{\text{pf}=6}$
            \begin{tabular}{|c|c|c|}
                \hline
                6 & 8 & 7\\
                \hline
                4 & 5 & 3\\
                \hline
            \end{tabular}
            $\xRightarrow{\text{pf}=7}$
            \begin{tabular}{|c|c|c|}
                \hline
                9 & 8 & 7\\
                \hline
                1 & 5 & 3\\
                \hline
            \end{tabular}
            $\xRightarrow{\text{pf}=8}$
            \begin{tabular}{|c|c|c|}
                \hline
                9 & 8 & 10\\
                \hline
                1 & 5 & 3\\
                \hline
            \end{tabular}
            $\xRightarrow{\text{pf}=8}$
            \begin{tabular}{|c|c|c|}
                \hline
                9 & 11 & 10\\
                \hline
                1 & 2 & 3\\
                \hline
            \end{tabular}
            $\xRightarrow{\text{pf}=9}$
            \begin{tabular}{|c|c|c|}
                \hline
                12 & 11 & 10\\
                \hline
                4 & 2 & 3\\
                \hline
            \end{tabular}
            $\xRightarrow{\text{pf}=10}$
            \begin{tabular}{|c|c|c|}
                \hline
                12 & 11 & 13\\
                \hline
                4 & 2 & 5\\
                \hline
            \end{tabular}
            $\xRightarrow{\text{pf}=11}$
            \begin{tabular}{|c|c|c|}
                \hline
                12 & 14 & 13\\
                \hline
                4 & 6 & 5\\
                \hline
            \end{tabular}
            $\xRightarrow{\text{pf}=12}$
            \begin{tabular}{|c|c|c|}
                \hline
                15 & 14 & 13\\
                \hline
                1 & 6 & 5\\
                \hline
            \end{tabular}
            $\xRightarrow{\text{pf}=13}$
            \begin{tabular}{|c|c|c|}
                \hline
                15 & 14 & 16\\
                \hline
                1 & 6 & 3\\
                \hline
            \end{tabular}
            $\xRightarrow{\text{pf}=14}$
            \begin{tabular}{|c|c|c|}
                \hline
                15 & 17 & 16\\
                \hline
                1 & 4 & 3\\
                \hline
            \end{tabular}
            $\xRightarrow{\text{pf}=15}$
            \begin{tabular}{|c|c|c|}
                \hline
                18 & 17 & 16\\
                \hline
                6 & 4 & 3\\
                \hline
            \end{tabular}
            $\xRightarrow{\text{pf}=16}$
            \begin{tabular}{|c|c|c|}
                \hline
                18 & 19 & 16\\
                \hline
                6 & 4 & 3\\
                \hline
            \end{tabular}
            $\xRightarrow{\text{pf}=16}$
            \begin{tabular}{|c|c|c|}
                \hline
                18 & 19 & 20\\
                \hline
                6 & 4 & 1\\
                \hline
            \end{tabular}
            $\xRightarrow{\text{pf}=17}$
            \begin{tabular}{|c|c|c|}
                \hline
                18 & 21 & 20\\
                \hline
                6 & 4 & 1\\
                \hline
            \end{tabular}
            $\xRightarrow{\text{pf}=17}$
            \begin{tabular}{|c|c|c|}
                \hline
                22 & 21 & 20\\
                \hline
                6 & 4 & 1\\
                \hline
            \end{tabular}
            $\xRightarrow{\text{pf}=17}$
            \begin{tabular}{|c|c|c|}
                \hline
                22 & 21 & 23\\
                \hline
                6 & 4 & 1\\
                \hline
            \end{tabular}
            $\xRightarrow{\text{pf}=17}$
            \begin{tabular}{|c|c|c|}
                \hline
                22 & 24 & 23\\
                \hline
                6 & 3 & 1\\
                \hline
            \end{tabular}
            $\xRightarrow{\text{pf}=18}$
            \begin{tabular}{|c|c|c|}
                \hline
                25 & 24 & 23\\
                \hline
                4 & 3 & 1\\
                \hline
            \end{tabular}
            $\xRightarrow{\text{pf}=19}$
            \begin{tabular}{|c|c|c|}
                \hline
                25 & 24 & 26\\
                \hline
                4 & 3 & 2\\
                \hline
            \end{tabular}
            $\xRightarrow{\text{pf}=20}$
            Termino con un totale di 20 page fault.
        \item \emph{Ideale}:\\
        Come dice il nome questo è un algoritmi perfetto, perchè dovrebbe prevedere il futuro, infatti dovrebbe sapere quali saranno le prossime richieste che un pogramma potrebbe fare.\\
        \begin{tabular}{|c|c|c|}
            \hline
            1 & null & null\\
            \hline
        \end{tabular}
        $\xRightarrow{\text{pf}=1}$
        \begin{tabular}{|c|c|c|}
            \hline
            1 & 3 & null\\
            \hline
        \end{tabular}
        $\xRightarrow{\text{pf}=2}$
        \begin{tabular}{|c|c|c|}
            \hline
            1 & 3 & 2\\
            \hline
        \end{tabular}
        $\xRightarrow{\text{pf}=3}$
        \begin{tabular}{|c|c|c|}
            \hline
            1 & 3 & 4\\
            \hline
        \end{tabular}
        $\xRightarrow{\text{pf}=4}$
        \begin{tabular}{|c|c|c|}
            \hline
            1 & 3 & 4\\
            \hline
        \end{tabular}
        $\xRightarrow{\text{pf}=4}$
        \begin{tabular}{|c|c|c|}
            \hline
            1 & 3 & 4\\
            \hline
        \end{tabular}
        $\xRightarrow{\text{pf}=4}$
        \begin{tabular}{|c|c|c|}
            \hline
            1 & 3 & 4\\
            \hline
        \end{tabular}
        $\xRightarrow{\text{pf}=4}$
        \begin{tabular}{|c|c|c|}
            \hline
            1 & 3 & 5\\
            \hline
        \end{tabular}
        $\xRightarrow{\text{pf}=5}$
        \begin{tabular}{|c|c|c|}
            \hline
            1 & 3 & 5\\
            \hline
        \end{tabular}
        $\xRightarrow{\text{pf}=5}$
        \begin{tabular}{|c|c|c|}
            \hline
            1 & 3 & 5\\
            \hline
        \end{tabular}
        $\xRightarrow{\text{pf}=5}$
        \begin{tabular}{|c|c|c|}
            \hline
            1 & 3 & 2\\
            \hline
        \end{tabular}
        $\xRightarrow{\text{pf}=6}$
        \begin{tabular}{|c|c|c|}
            \hline
            1 & 3 & 4\\
            \hline
        \end{tabular}
        $\xRightarrow{\text{pf}=7}$
        \begin{tabular}{|c|c|c|}
            \hline
            1 & 5 & 4\\
            \hline
        \end{tabular}
        $\xRightarrow{\text{pf}=8}$
        \begin{tabular}{|c|c|c|}
            \hline
            1 & 6 & 4\\
            \hline
        \end{tabular}
        $\xRightarrow{\text{pf}=9}$
        \begin{tabular}{|c|c|c|}
            \hline
            1 & 6 & 4\\
            \hline
        \end{tabular}
        $\xRightarrow{\text{pf}=9}$
        \begin{tabular}{|c|c|c|}
            \hline
            3 & 6 & 4\\
            \hline
        \end{tabular}
        $\xRightarrow{\text{pf}=10}$
        \begin{tabular}{|c|c|c|}
            \hline
            3 & 6 & 4\\
            \hline
        \end{tabular}
        $\xRightarrow{\text{pf}=10}$
        \begin{tabular}{|c|c|c|}
            \hline
            3 & 6 & 4\\
            \hline
        \end{tabular}
        $\xRightarrow{\text{pf}=10}$
        \begin{tabular}{|c|c|c|}
            \hline
            3 & 6 & 4\\
            \hline
        \end{tabular}
        $\xRightarrow{\text{pf}=10}$
        \begin{tabular}{|c|c|c|}
            \hline
            1 & 6 & 4\\
            \hline
        \end{tabular}
        $\xRightarrow{\text{pf}=11}$
        \begin{tabular}{|c|c|c|}
            \hline
            1 & 6 & 4\\
            \hline
        \end{tabular}
        $\xRightarrow{\text{pf}=11}$
        \begin{tabular}{|c|c|c|}
            \hline
            1 & 6 & 4\\
            \hline
        \end{tabular}
        $\xRightarrow{\text{pf}=11}$
        \begin{tabular}{|c|c|c|}
            \hline
            1 & 6 & 4\\
            \hline
        \end{tabular}
        $\xRightarrow{\text{pf}=11}$
        \begin{tabular}{|c|c|c|}
            \hline
            3 & 6 & 4\\
            \hline
        \end{tabular}
        $\xRightarrow{\text{pf}=12}$
        \begin{tabular}{|c|c|c|}
            \hline
            3 & 6 & 4\\
            \hline
        \end{tabular}
        $\xRightarrow{\text{pf}=12}$
        \begin{tabular}{|c|c|c|}
            \hline
            2 & 6 & 4\\
            \hline
        \end{tabular}
        $\xRightarrow{\text{pf}=13}$
        Termino con un totale di 13 page fault.
    \end{itemize}
    \section*{Domanda 5}
    \textbf{MAX PUNTI:} 8\\
    \textbf{Domanda:}


    Si ipotizzi di avere una toilette pubblica unisex che viene gestita secondo i seguenti punti:\\
    \begin{enumerate}
        \item La toilette può essere libera, occupata da donnae, occupata da uomini.
        \item Se è libera la persona che arriva la occupa.
        \item Se la toilette è occupata da una o più donne, possono entrare solo donne.
        \item Se la toilette è occupata da uno o più uomini, possono entrare solo uomini.
    \end{enumerate}
    Progettare un sistema per la sincronizzazione con il metodo che si vuole.\\
    Non fare alcuna assunzione sul numero di uomini/donne.\\
    \textbf{Soluzione:}
    \begin{center}
        \emph{Variabili in comune}
        \begin{lstlisting}
            Toilette wc; //Risorsa da usare
            List<Uomo> codaMaschile; //Coda attesa maschile
            List<Donna> codaFemminile; //Coda attesa femminile
            Monitor monitor=free;  //Inizialmente il monitor segna la toilette come libera
        \end{lstlisting}
    \end{center}
    \begin{lstlisting}
        Uomo(){
            while(true){
                codaMaschile.append(new Uomo());
                if(monitor.isFree()){
                    monitor.lock();
                    while(codaMaschile.isNotEmpty()){
                        Uomo tmp=codaMaschile.remove();
                        tmp.use(wc);
                    }
                    monitor.free();
                }
            }
        }
    \end{lstlisting}
    \begin{lstlisting}
        Donna(){
            while(true){
                codaFemminile.append(new Donna());
                if(monitor.isFree()){
                    monitor.lock();
                    while(codaFemminile.isNotEmpty()){
                        Uomo tmp=codaFemminile.remove();
                        tmp.use(wc);
                    }
                    monitor.free();
                }
            }
        }
    \end{lstlisting}
\end{document}