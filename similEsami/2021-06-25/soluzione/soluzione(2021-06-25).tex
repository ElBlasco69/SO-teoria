\documentclass{article}
\usepackage[margin=1.5cm]{geometry}
\usepackage{mathtools}
\usepackage{amsfonts}
\usepackage{listings}
\lstset{
  basicstyle=\ttfamily,
  mathescape
}
\title{Esame 2021/06/25}
\author{Blascovich Alessio}
\date{}

\begin{document}
   \maketitle
   \section*{Domanda 1}
   \textbf{MAX PUNTI:} 6\\
   \textbf{Domanda:}


   Dare la definizione di binding degli indirizzi e spiegare le tipologie di collegamento(linking) e caricamento(loading).\\
   \textbf{Soluzione:}


   Per binding degli indirizzi si definisce la traduzione degli indirizzi logici usati da un problema a indirizzi fisici usati da quel programma una volta che viene trasformato in processo, il binding può essere fatto in diverse fasi:
   \begin{itemize}
      \item \emph{Compile time}: richiede di conoscere precedentemente le locazioni di memeoria nelle quali verrà caricato il programma, se le locazioni di memoria variano allora bisogna ricompilare il programma.
      \item \emph{Load time}: generazione degli indirzzi in base alla prima locazione di memoria occupata dal programma, di conseguenza gli indirizzi saranno della forma indirizzo\_inizio\_programma+offset.\\
      Però se il programma viene sspostato devo ricompilare il programma.
      \item \emph{Run time}: il binding è posticipato per permettere al processo di essere spostato in memoria durante la sua esecuzione, questo metodo richiede però supporto hardware aggiuntivo.
   \end{itemize}
   Il \emph{linking} può avvenire in due modi diversi:
   \begin{enumerate}
      \item \emph{Statico}: durante la compilazione viene caricata un copia intera delle librerie usate, quindi tutti i riferimenti sono risolti durante la compilazione.
      \item \emph{Dinamico}: il codice in se contiene dei puntatori(stub) alla funzione esterna chiamata.
      Un esempio di questo metodo sono i file .ddl(\emph{Dynamic Linked Library}) usate da Windows.
   \end{enumerate}
   Il \emph{loading} può essere effettuato in due modi diversi:
   \begin{enumerate}
      \item \emph{Statico}: l'intero programma viene caricato per intero in memoria.
      \item \emph{Dinamico}: il processo viene diviso in sub routine così da caricarle in memoria solo quando necessario.
   \end{enumerate}
   \section*{Domanda 2}
   \textbf{MAX PUNTI:} 6\\
   \textbf{Domanda:}


   Si consideri un sistema con 3 processi $P_1$, $P_2$ e $P_3$ con 3 tipi di risorse $A$, $B$ e $C$.\\
   Si supponga che all'istante $T$ il sistema si trovi nella seguente condizione:


   \begin{minipage}[c]{0.5\textwidth}
      \emph{ALLOC}\\
      \begin{tabular}{|c|c|c|c|}
         \hline
         \emph{processo} & $A$ & $B$ & $C$\\
         \hline
         $P_1$ & 2 & 2 & 3\\
         \hline
         $P_2$ & 2 & 0 & 3\\
         \hline
         $P_3$ & 1 & 2 & 4\\
         \hline 
      \end{tabular}
   \end{minipage}
   \begin{minipage}[c]{0.5\textwidth}
      \emph{MAX}\\
      \begin{tabular}{|c|c|c|c|}
         \hline
         \emph{processo} & $A$ & $B$ & $C$\\
         \hline
         $P_1$ & 3 & 6 & 8\\
         \hline
         $P_2$ & 4 & 3 & 3\\
         \hline
         $P_3$ & 3 & 4 & 4\\
         \hline 
      \end{tabular}
   \end{minipage}\\
   Si supponga che allo stesso istante $T$ siano ancora disponibli 2 di $A$, 3 di $B$ e 0 di $C$.
   \begin{itemize}
      \item[a] Il sisetma è in stato safe?
      \item[b] All'istante $T$ il sistema può soddisfare la richiesta $P_1(1,0,1)$?
      \item[c] All'istante $T$ il sistema può soddisfare la richiesta $P_1(2,0,0)$?
   \end{itemize}
   \textbf{Soluzione:}
   \begin{itemize}
      \item[a] Per verificare se il sistema è in uno stato safe uso l'algortimo del banchiere senza la parte che considera una possibile futura richiesta.
         L'algortimo del banchiere crea da prima una tabella \emph{Nedd} definita come \emph{Nedd}[]=\emph{MAX}[]-\emph{ALLOC}[].\\
         Devo inoltre porre il vettore \emph{work}[]=(2,3,0).
         \begin{center}
            \emph{Need}\\
            \begin{tabular}{|c|c|c|c|}
               \hline
               \emph{processo} & $A$ & $B$ & $C$\\
               \hline
               $P_1$ & 1 & 4 & 5\\
               \hline
               $P_2$ & 2 & 3 & 0\\
               \hline
               $P_3$ & 2 & 2 & 0\\
               \hline
            \end{tabular}
         \end{center}
         Adesso devo verificare di essere in uno stato safe con i seguenti passaggi:
         \begin{enumerate}
            \item Scorro gli elementi di \emph{Need} fino a trovare una righa tale che \emph{Need}[] $\le$ \emph{Work}[].
            \item Trovata la riga eseguo \emph{Work}[]=\emph{Work}[]+\emph{ALLOC}[$P_i$], torno al punto 1 escludendo dalle prossime valutazioni la riga nella quale mi sono  precedentemente fermato.
            \item Se arrivo fino alla fine senza aver trovato una riga \emph{Need}[] $\le$ \emph{Work}[] allora \textbf{non} sono in uno stato safe.
         \end{enumerate}
         Inizio a iterare l'algoritmo.
         \begin{itemize}
            \item prima iterazione\\
               (1,4,5) $\le$ (2,3,0) false $\Rightarrow$ ++i\\
               (2,3,0) $\le$ (2,3,0) true $\Rightarrow$ \emph{Work}=\emph{Work}+\emph{ALLOC}[$P_B$]=(4,3,3).
            \item seconda iterazione\\
               (1,4,5) $\le$ (4,3,3) false $\Rightarrow$ ++i\\
               (2,2,0) $\le$ (4,3,3) true $\Rightarrow$ \emph{Work}=\emph{Work}+\emph{ALLOC}[$P_C$]=(5,5,7).
            \item terza iterazione\\
               (1,4,5) $\le$ (5,5,7) true $\Rightarrow$ \emph{Work}=\emph{Work}+\emph{ALLOC}[$P_A$]=(7,7,10).
         \end{itemize}
         Ho concluso tutte le iterazioni servendo per ogni iterazione un processo, quindi sono in uno stato safe.\\
      \item[b] Ora simulo la richiesta $P_1(1,0,1)$=\emph{Req}.\\
         Devo verificare che $\forall i$ vale che \emph{Req}[]$\le$\emph{Need}[$i$][] e \emph{Req}[]$\le$\emph{Work}[].
         \begin{itemize}
            \item \emph{Req}[]$\le$\emph{Need}[1][] $\Rightarrow$ (1,0,1)$\le$(1,4,5) true.
            \item \emph{Req}[]$\le$\emph{Need}[2][] $\Rightarrow$ (1,0,1)$\le$(2,3,0) false.
         \end{itemize}
         Essendo un requisito non soddisfatta mi fermo subito.
      \item[c] Ora simulo la richiesta $P_1(2,0,0)$=\emph{Req}.\\
         Devo verificare che $\forall i$ vale che \emph{Req}[]$\le$\emph{Need}[$i$][] e \emph{Req}[]$\le$\emph{Work}[].
         \begin{itemize}
            \item \emph{Req}[]$\le$\emph{Need}[1][] $\Rightarrow$ (2,0,0)$\le$(1,4,5) false.
         \end{itemize}
         Essendo un requisito non soddisfatta mi fermo subito.
   \end{itemize}
   \section*{Domanda 3}
   \textbf{MAX PUNTI:} 6\\
   \textbf{Domanda:}


   Descrivere nel dettaglio il funzionamento del RAID di livello 5.\\
   \textbf{Soluzione:}


   Il RAID di 5° livello gestisce i dischi a livello di blocchi, i dati di ogni disco sono distribuiti tra tutti i dischi, e così via.\\
   Ogni disco al suo interno ha il blocco che contiene le parità di un altro disco del RAID.\\
   Infatti al disco $A$ è proibito avere al suo interno il blocco con le proprie parità.\\
   E' simile al RAID di 4° livello solo che risolve il problema del singolo disco che contiene tutti i bit di parità, però la scrittura sui singoli dischi rimane comunque molto lenta come nel RAID di 4° livello.
   \section*{Domanda 4}
   \textbf{MAX PUNTI:} 6\\
   \textbf{Domanda:}


   Descrivere le principali differenze tra un processo ed un thread, facendo l'esempio di un caso pratico in cui conviene usare un thread anzichè un processo.\\
   \textbf{Soluzione:}
   Un processo è associato con uno spazio di indirizzi e a delle risorse del sistema, per questo quando io creo un processo figlio implicitamente creo una copia di tutte le variabili del processo padre e alloco delle nuove risorse al figlio.\\
   Un thread è associato ai valori dei registri, al program counter(PC) al suo stato di esecuzione e allo stato dello stack.\\
   Infatti un thread condivide lo spazio degli indirizzi con il processo da cui è stato generato e anche le risorse allocate al processo padre.\\
   Un thread ha molti vantaggi:
   \begin{itemize}
      \item Riducono il tempo di risposta perchè il thread non è bloccante.
      \item Ha una comunicazione aggevolata col padre perchè condivide con esso lo spazio degli indirizzi.
      \item La Creazione/distruzione/switch di thread è molto più veloce che tra singoli processi.
      \item I thread sono molto scalabili, infatti aumentano il livello di parallelismo dei processi se la macchina è multiprocessore.
   \end{itemize}
   A livello pratico multissimi programmi usano i thread per esempio grazie ai thread io posso navigare su internet mentre ascolto il contenuto di un file audio o scrivere un file di testo mentre ho aperto un riproduttore di video.
   \section*{Domanda 5}
   \textbf{MAX PUNTI:} 8\\
   \textbf{Domanda:}


   Dato un calcolatore con due gruppi di processi ed un buffer il quale è condiviso tra i due gruppi.\\
   Un insieme di processi detto \verb+Reader+ è consentito accedere al buffer solo in modo mutualmente esclusivo con altri processi dello stesso gruppo e con processi del gruppo \verb+Writer+.\\
   Scrivere uno pseudocodice che permette l'accesso al buffer dando precedenza ai processi del gruppo \verb+Reader+, quindi un \verb+Reader+ deve attendere solo e solo se un \verb+Writer+ è già in esecuzione.\\
   Finchè arrivano \verb+Reader+ i \verb+Writer+ devono attendere.\\
   \textbf{Soluzione:}
   \begin{lstlisting}
      Semaforo_bin usoBuffer=1;
      Semaforo_bin usoPrecedenza=1;
      bool precedenza=false;

      while(true){
         Reader(){
            P(usoPrecedenza);
            precedenza=true;
            V(usoPrecedenza);

            P(usoBuffer);
            //modifica buffer
            V(usoBuffer);

            P(usoPrecedenza);
            precedenza=false;
            V(usoPrecedenza);
         }

         Writer(){
            P(usoPrecedenza);
            if(precedenza==true){
               V(usoPrecedenza);
            }
            else{
               P(usoBuffer);
               //modifica buffer
               V(usoBuffer)
            }
         }
      }
   \end{lstlisting}
\end{document}